%-load-%
\def\load#1 {%
  \expandafter\ifx\csname#1\endcsname\relax%
    \expandafter\def\csname#1\endcsname{}%
    \input #1\relax%
  \fi}%

%-dependencies-%
\load documentation
\load controlsequence

%-documentation-%
\documentation{load}{This package provides the control sequence {\cs\\load} which is designed work like {\cs\\input}. The sole difference is, that {\cs\\load} tries not to load packages multiple times. For this purpose a controll sequence of the name of the package is created, when it is first loaded. Later when you try to load the package again this control sequence is found and the call can thus be ignored. To see how this could be useful besides taking strain of the compiler, imagine you had a package for documenting other packages which uses a package you also want to document. Using {\cs\\load} these packages can both be {\cs\\load}ed into each other without having the problem of a loop on compilation.}
