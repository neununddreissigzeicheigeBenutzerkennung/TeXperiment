%-registers-%
\newcount\bulletcount%
\newdimen\bulletswidth%
\newdimen\transcendedimen%

%-list-%
\long\def\list#1{{%
  \expandafter\ifx\csname lbullet\endcsname\relax%
    \def\lbullet{\advance\bulletcount1\relax\number\bulletcount.\hskip\fontdimen2\the\font\hskip\fontdimen7\the\font}\fi%
  \expandafter\ifx\csname newlist\endcsname\relax%
    \def\newlist{\par\bulletcount0}\fi%
  \expandafter\ifx\csname itembreak\endcsname\relax%
    \def\itembreak{\par}\fi%TODO make \break possible
  \newlist%
  \setbox0\vbox{
    \long\def\item##1{%
      \noindent%
      \setbox0\hbox{%
        \lbullet}%
      \ifdim\wd0>\bulletswidth%
        \bulletswidth\wd0\fi%
      \lbullet%
      ##1%
      \itembreak}
    \bulletswidth0pt%
    #1%
    \global\transcendedimen\bulletswidth}%
  \bulletswidth\transcendedimen%
  \long\def\item##1{%
  \advance\leftskip\bulletswidth%
    \noindent%
    \vrule width0pt\hskip-\bulletswidth%
    \setbox0\hbox{\lbullet}%
    \lbullet%
    \hskip-\wd0%
    \hskip\bulletswidth%
    ##1%
    \itembreak%
  \advance\leftskip-\bulletswidth}%
  #1}}

%-dependencies-%
\input load
\load documentation

%-documentation-%
\documentation{alist}{This package provides the control sequence {\cs\\list\#1\{\\item\#\#1\}} which can be used for typesetting lists.
\list{
  \item{The list environment is relatively simple to use.}
  \item{It's bullet points may be defined to be anything you want. By default they count the itemns using natural numbers. Bullet points are assigned by defining the control sequence {\cs\\lbullet}}
  {\def\lbullet{$\circ\quad$}\item{Bullet points may also be redefined in any local group. This makes them very versitile.}}
  \item{Lists may be nested inside of each other\list{
    \item{To reset the {\cs\\count} for numbering items automatically or simmilar things the control sequence {\cs\\newlist} is executed, whenever a new list begins.}
    \item{{\cs\\newlist} may be redifined anywhere simmilar to {\cs\\lbullet}}
    \item{By default {\cs\\newlist} adds a {\cs\\par} and resets the number.}}}
  \item{If a {\cs\\item}'s content spans over multiple lines you will notice, that the Text is aligned behind the {\cs\\item}'s {\cs\\lbullet}. To avoid unwanted grouping this is achieved by simply changing the {\cs\\leftskip} dimension.}
  \item{To have the {\cs\\lbullet}'s ``sticking out'' at the left an {\cs\\hskip} is inserted. This would obviously cause trouble if two {\cs\\lbullet}'s were to immediatly follow each other.}
  \item{Such problems are usually avoided by the {\cs\\par} in {\cs\\newlist}. If you want to change this behaviour I wish you good luck.}
  \item{A better alternative to achive whatever you want to mess with {\cs\\newlist} for may be to just add a new {\cs\\list} inbetween two items and to change the {\cs\\lbullet} to whatever is required.}
  \item{You can manually change the {\cs\\leftskip} for this. You might want to access the {\cs\\dimen} {\cs\\bulletswidth} which is used by {\cs\\list} to store the width of the widest {\cs\\lbullet}.}
  {\advance\leftskip\bulletswidth\list{
    \item{As you can see, this is not a very good example of what I just described.}
    \item{You will just have to believe me, that this is diffrent from {\cs\\item} 3.}}}
  \item{One of the problem with spotting what is diffrent about my previous example is, that {\cs\\item}'s are seperated by the same {\cs\\par} that is inserted before a list. While this can be changed by altering both {\cs\\newlist} or {\cs\\itembreak}, it is necessary for {\cs\\itembreak } to contain {\cs\\par} at least indirectly.}
  \item{Now that we have reached double digits. I can stop wasting your time with bad examples for you can now observe how {\cs\\list} automatically takes care of providing the approriate space for bullets even if they are of diffrent lenth.}
  {\def\lbullet{$\circ\quad$}
  \item{Just to remove all doubt}}}}
