%-parameters-%
  \newcount\digitrange%
  \newcount\maximumdigit%
  \newcount\digitvalue%
  \newcount\nextparameter%
  \newcount\originalvalue%
%-customnumeral-%
  \def\customnumeral#1#2#3#4{{%
      \ifnum#1>#2%
          \digitrange#1\relax%
          \maximumdigit#2\relax%
      \else%
          \digitrange#2\relax%
          \maximumdigit#1\relax\fi%
      \originalvalue#3\relax%
      \digitvalue\originalvalue\relax%
      \ifnum\digitvalue>\maximumdigit%
          \divide\digitvalue\digitrange%
          \nextparameter=\digitvalue%
          \multiply\digitvalue-\digitrange%
          \advance\digitvalue\originalvalue\relax%
          \ifnum\digitvalue>\maximumdigit%
              \advance\digitvalue-\digitrange%
              \advance\nextparameter1\fi%
          \customnumeral\digitrange\maximumdigit\nextparameter{#4}\fi%
      #4\relax}}%
%-alphanumeral-%
  \def\alphanumeral#1{%
      \customnumeral{26}{25}{#1}{%
          \advance\digitvalue97\relax%
          \char\digitvalue}}%
%-Alphanumeral-%
  \def\Alphanumeral#1{%
      \customnumeral{26}{25}{#1}{%
          \advance\digitvalue65\relax%
          \char\digitvalue}}%
%-sillynumeral-%
  \def\simmilarnumeral#1{%
       \customnumeral{1}{3}{#1}{%
           \number\digitvalue}}%
%-dependencies-%
  \input load %
  \load documentation %
%-documentation-%
  \documentation{customnumeral}{%
      This package contains the control sequences
      {\cs\\customnumeral\#1\#2\#3\#4}, {\cs\\alphanumeral\#1},
      {\cs\\Alphanumeral\#1} and {\cs\\sillynuneral\#1}.  It also uses
      the {\cs\\count}'s {\cs\\digitrange}, {\cs\\maximumdigit},
      {\cs\\digitvalue}, {\cs\\nextparameter} and
      {\cs\\originalvalue}.  \par The main control sequence
      {\cs\\customnumeral\#1\#2\#3\#4} can be used to display whole
      number in custom numeral systems.  The {\cs\#1} argument takes
      maximum value a digit can hold.  {\cs\#2} contains the size of
      the range of values a digit can hold.  {\cs\#3} contains the
      value to be displayrd in the numeral system. {\cs\#4} contains a
      sequence executed to output each digit. The {\cs\\count}
      {\cs\\digitvalue} can be used to obtain the value of a number to
      specify how to display it.  \par The control sequence
      {\cs\\alphanumeral\#1} displays numbers with alphanumeric
      digits.  The digit a represents the value 0, b 1, \dots, z
      25.  \par The control sequence {\cs\\Alphanumeral\#1} is verry
      simmilar to {\cs\\alphanumeral\#1}.  The only difference is,
      that the digits are in uppercase.  \par The control sequence
      {\cs\\sillynumeral\#1} displays numbers with the digits -1, 0,
      1.}
